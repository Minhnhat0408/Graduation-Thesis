\section{Đặc tả yêu cầu}
Phần này trình bày các yêu cầu chức năng và phi chức năng cho hệ thống VieVu, được xác định dựa trên mục tiêu dự án và nhu cầu người dùng. Việc đặc tả rõ ràng các yêu cầu này là nền tảng cho quá trình thiết kế, triển khai và kiểm thử hệ thống.
\subsection{Yêu cầu chức năng}

Để hệ thống VieVu hoạt động hiệu quả, đáp ứng nhu cầu kết nối, lập kế hoạch và chia sẻ trải nghiệm du lịch nhóm một cách thông minh và tiện lợi, các yêu cầu chức năng chính được xác định như sau:

\begin{itemize}
    \item[-] \textbf{Đăng ký và xác thực người dùng:} Cho phép đăng ký tài khoản mới và đăng nhập, bao gồm xác thực thông tin.
    \item[-] \textbf{Quản lý hồ sơ người dùng:} Cho phép người dùng xem, cập nhật hồ sơ cá nhân và xem hồ sơ người dùng khác.
    \item[-] \textbf{Tra cứu Thông tin Du lịch:} Cho phép tìm kiếm, lọc và xem chi tiết địa điểm/sự kiện du lịch (danh sách/bản đồ).
    \item[-] \textbf{Quản lý chuyến đi:} Cho phép tạo, sửa, hủy chuyến đi; xem và tham gia chuyến đi công khai.
    \item[-] \textbf{Quản lý thành viên chuyến đi:} Cho phép chủ chuyến đi mời, xóa, ủy quyền thành viên và quản lý quyền truy cập.
    \item[-] \textbf{Quản lý lịch trình chuyến đi:} Cho phép tạo, sửa, xóa mục lịch trình; xem lịch trình (danh sách/bản đồ) và đánh dấu hoàn thành.
    \item[-] \textbf{Lưu các địa điểm yêu thích:} Hệ thống cho phép người dùng lưu lại các địa điểm yêu thích của mình vào chuyến đi mà họ tham gia để dễ dàng truy cập và dùng nguyên liệu tạo lịch trình.
    \item[-] \textbf{Nhắn tin:} Cung cấp chat riêng và chat nhóm trong chuyến đi để trao đổi thông tin.
    \item[-] \textbf{Tổng hợp lịch trình từ chat:} Tự động phân tích và tóm tắt lịch trình nháp từ nội dung chat nhóm khi được yêu cầu.
    \item[-] \textbf{Nhận diện địa điểm trong tin nhắn:} Tự động nhận diện và làm nổi bật tên địa điểm trong chat, cho phép xem nhanh thông tin.
    \item[-] \textbf{Bỏ phiếu lịch trình trong tin nhắn:} Cho phép thành viên bỏ phiếu (reaction) cho các đề xuất lịch trình trong chat.
    \item[-] \textbf{Tự động cập nhật trạng thái chuyến đi:} Tự động thay đổi trạng thái chuyến đi (Sắp diễn ra, Đang diễn ra, Đã kết thúc) dựa trên thời gian.
    \item[-] \textbf{Chia sẻ vị trí thời gian thực:} Cho phép thành viên chia sẻ vị trí hiện tại với nhóm trong chuyến đi đang diễn ra.
    \item[-] \textbf{Đánh giá sau chuyến đi:} Cho phép người dùng đăng đánh giá về chuyến đi và đánh giá lẫn nhau sau khi kết thúc.
    \item[-] \textbf{Thông báo Hệ thống:} Cung cấp cơ chế gửi thông báo về các sự kiện quan trọng (lời mời, tin nhắn mới, thay đổi trạng thái, nhắc lịch trình).
\item[-] \textbf{Hệ thống Gợi ý:} Tích hợp gợi ý thông minh để đề xuất nội dung phù hợp (địa điểm cá nhân hóa dựa trên khảo sát/hành vi, địa điểm liên quan).
\end{itemize}

\subsection{Yêu cầu phi chức năng}

Hệ thống cần đáp ứng các yêu cầu về chất lượng và ràng buộc sau:

\begin{itemize}
    \item[-] \textbf{Hiệu năng:} Phản hồi giao diện nhanh; tải dữ liệu ban đầu < 3s; độ tác vụ thời gian thực < 1s; xử lý AI < 10-15s.
    \item[-] \textbf{Bảo mật và Toàn vẹn Dữ liệu:} Bảo vệ thông tin cá nhân, vị trí, chat; mã hóa dữ liệu nhạy cảm; xác thực mạnh; phân quyền chi tiết (RLS); đảm bảo toàn vẹn dữ liệu.
    \item[-] \textbf{Độ chính xác và Tin cậy của AI:} NER nhận diện đúng địa điểm; ZSC lọc đúng tin nhắn kế hoạch; tóm tắt lịch trình có ý nghĩa; gợi ý phù hợp và hữu ích.
    \item[-] \textbf{Khả năng mở rộng:} Kiến trúc backend và cơ sở dữ liệu có khả năng mở rộng để xử lý tải tăng cao mà không giảm hiệu suất.
    \item[-] \textbf{Tính khả dụng và Dễ sử dụng:} Backend có độ sẵn sàng cao (> 99.5\%); giao diện trực quan, thẩm mỹ, dễ sử dụng.
    \item[-] \textbf{Tối ưu Tài nguyên Thiết bị:} Ứng dụng di động hoạt động mượt mà, tiết kiệm CPU, RAM, và pin, đặc biệt khi chạy nền hoặc chia sẻ vị trí.
    \item[-] \textbf{Tính tương thích:} Hoạt động ổn định trên nhiều thiết bị và phiên bản HĐH phổ biến (Android 8.0+, iOS 14+), hiển thị tốt trên các kích thước màn hình.
    \item[-] \textbf{Cập nhật Thời gian thực Mượt mà:} Thông tin thời gian thực (chat, vị trí, trạng thái) được cập nhật nhanh chóng, mượt mà trên giao diện.
\end{itemize}
\section{Đặc tả yêu cầu}
Phần này trình bày các yêu cầu chức năng và phi chức năng cho hệ thống VieVu, được xác định dựa trên mục tiêu dự án và nhu cầu người dùng. Việc đặc tả rõ ràng các yêu cầu này là nền tảng cho quá trình thiết kế, triển khai và kiểm thử hệ thống.
\subsection{Yêu cầu chức năng}

Để hệ thống VieVu hoạt động hiệu quả, đáp ứng nhu cầu kết nối, lập kế hoạch và chia sẻ trải nghiệm du lịch nhóm một cách thông minh và tiện lợi, các yêu cầu chức năng chính được xác định như sau:

\begin{itemize}
    \item[-] \textbf{Đăng ký và xác thực người dùng:} Hệ thống cho phép đăng ký tài khoản mới và đăng nhập, bao gồm xác thực thông tin.
    \item[-] \textbf{Quản lý hồ sơ người dùng:} Hệ thống cho phép người dùng xem, cập nhật hồ sơ cá nhân và xem hồ sơ người dùng khác.
    \item[-] \textbf{Tra cứu Thông tin Du lịch:} Hệ thống cho phép tìm kiếm, lọc và xem chi tiết địa điểm,sự kiện du lịch dưới dạng danh sách và bản đồ trực quan.
    \item[-] \textbf{Quản lý chuyến đi:} Hệ thống cho phép người dùng tạo, sửa, xóa chuyến đi của cá nhân và xem, tham gia chuyến đi công khai của các thành viên khác. Trong chuyến đi, người dùng sẽ quản lý thành viên (mời, xóa, ủy quyền, quản lý quyền truy cập), quản lý lịch trình (tạo, sửa, xóa mục, xem danh sách/bản đồ, đánh dấu hoàn thành) và lưu các địa điểm ưa thích để thêm vào lịch trình.
    % \item[-] \textbf{Quản lý thành viên chuyến đi:} Hệ thống cho phép chủ chuyến đi mời, xóa, ủy quyền thành viên và quản lý quyền truy cập.
    % \item[-] \textbf{Quản lý lịch trình chuyến đi:} Hệ thống cho phép tạo, sửa, xóa mục lịch trình; xem lịch trình (danh sách/bản đồ) và đánh dấu hoàn thành.
    % \item[-] \textbf{Lưu các địa điểm yêu thích:} Hệ thống Hệ thống cho phép người dùng lưu lại các địa điểm yêu thích của mình vào chuyến đi mà họ tham gia để dễ dàng truy cập và dùng nguyên liệu tạo lịch trình.
    \item[-] \textbf{Nhắn tin:} Hệ thống cho phép người dùng nhắn tin riêng hoặc nhắn tin theo nhóm từ chuyến đi.
    \item[-] \textbf{Tổng hợp lịch trình từ chat:} Hệ thống phân tích và tóm tắt lịch trình nháp từ nội dung chat nhóm khi được người dùng yêu cầu.
    \item[-] \textbf{Nhận diện địa điểm trong tin nhắn:} Hệ thống yự động nhận diện và làm nổi bật tên địa điểm trong chat, cho phép xem nhanh thông tin về địa điểm đó.
    \item[-] \textbf{Bỏ phiếu lịch trình trong tin nhắn:} Hệ thống cho phép thành viên trong chuyến đi bỏ phiếu các ý kiến thông qua reaction cho các đề xuất lịch trình trong tin nhắn.
    \item[-] \textbf{Tự động cập nhật trạng thái chuyến đi:} Hệ thống tự động cập nhật trạng thái chuyến đi (Sắp diễn ra, Đang diễn ra, Đã kết thúc) dựa trên thời gian thực tế.
    \item[-] \textbf{Chia sẻ vị trí thời gian thực:}  Hệ thống cho phép các thành viên trong chuyến đi chia sẻ vị trí hiện tại với nhau khi chuyến đi đang diễn ra.
    \item[-] \textbf{Đánh giá sau chuyến đi:}  Hệ thống cho phép người dùng đăng đánh giá về chuyến đi và đánh giá lẫn nhau sau khi kết thúc chuyến đi.
    \item[-] \textbf{Thông báo:}  Hệ thống cung cấp cơ chế gửi thông báo đẩy và thông báo trong ứng dụng về các sự kiện quan trọng (lời mời, tin nhắn mới, thay đổi trạng thái, nhắc lịch trình).
    \item[-] \textbf{Gợi ý:}  Hệ thống tích hợp gợi ý thông minh để đề xuất nội dung phù hợp (địa điểm cá nhân hóa dựa trên khảo sát/hành vi, địa điểm liên quan).
\end{itemize}

\subsection{Yêu cầu phi chức năng}

Hệ thống cần đáp ứng các yêu cầu về chất lượng và ràng buộc sau:

\begin{itemize}
    \item[-] \textbf{Hiệu năng:} Hệ thống cần đảm bảo phản hồi giao diện nhanh, tải dữ liệu ban đầu dưới 3 giây. Tính năng thời gian thực (chat, vị trí) có độ trễ thấp (<2s). Xử lý AI hợp lý (<10-15s).

    \item[-] \textbf{Bảo mật và Toàn vẹn Dữ liệu:} Hệ thống cần ưu tiên bảo vệ thông tin người dùng (cá nhân, vị trí, chat). Áp dụng mã hóa, xác thực mạnh, phân quyền chi tiết để ngăn truy cập trái phép và đảm bảo toàn vẹn dữ liệu.

    \item[-] \textbf{Độ chính xác và Tin cậy AI:} Các thuật toán AI (NER, ZSC, tóm tắt, gợi ý) cần hoạt động hiệu quả, đáng tin cậy và cung cấp kết quả hữu ích.

    \item[-] \textbf{Khả năng mở rộng:} Kiến trúc hệ thống (DB, backend) phải có khả năng mở rộng để xử lý lượng người dùng, chuyến đi và dữ liệu tăng lên mà không giảm hiệu suất.

    \item[-] \textbf{Tính khả dụng và Dễ sử dụng:} Dịch vụ backend cần có độ sẵn sàng cao (>99.5\%). Giao diện người dùng phải trực quan, dễ hiểu và dễ thao tác.

    \item[-] \textbf{Tối ưu Tài nguyên Thiết bị:} Ứng dụng di động cần tối ưu sử dụng CPU, RAM và tiết kiệm pin, đặc biệt khi chạy nền hoặc chia sẻ vị trí.

    \item[-] \textbf{Tính tương thích:} Hệ thống cần đảm bảo hoạt động ổn định trên nhiều thiết bị, phiên bản hệ điều hành (Android 8.0+, iOS 14+) và kích thước màn hình.

    \item[-] \textbf{Cập nhật Thời gian thực Mượt mà:} Thông tin thời gian thực (tin nhắn, vị trí, trạng thái) cần được cập nhật nhanh chóng, gần như tức thì và mượt mà trên giao diện.
\end{itemize}
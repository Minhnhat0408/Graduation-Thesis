\subsection{Ca sử dụng gửi tin nhắn}
\noindent Ca sử dụng này mô tả cách người dùng gửi tin nhắn trong các cuộc hội thoại nhóm của chuyến đi hoặc các cuộc hội thoại riêng tư. Hệ thống cũng thực hiện nhận dạng địa điểm trong tin nhắn và phân loại tin nhắn để hỗ trợ tổng hợp lịch trình. Bảng~\ref{tab:uc_send_message_spec} trình bày chi tiết đặc tả ca sử dụng, bao gồm luồng sự kiện chính, luồng thay thế, các điều kiện và yêu cầu liên quan. Các biểu đồ hoạt động, quan hệ (Bảng~\ref{tab:uc_send_message_diagrams}) và tuần tự (Hình~\ref{fig:3-3-9-sequence-diagram}) minh họa rõ hơn về quy trình và tương tác hệ thống khi người dùng gửi tin nhắn.
% \vspace{0.5cm} % Adjust spacing if needed

% Use longtable environment
% Need \usepackage{longtable} and \usepackage{calc} in preamble
\begin{longtable}{| p{4cm} | p{\dimexpr\linewidth-4cm-4\tabcolsep} |} % Adjust widths as needed
    \caption{Đặc tả ca sử dụng gửi tin nhắn} % Caption inside longtable
    \label{tab:uc_send_message_spec} \\ % Label after caption

    \hline
    \textbf{Mô tả} & Người dùng có thể gửi tin nhắn trong nhóm hoặc gửi tin nhắn riêng cho bạn bè. \\
    \hline
    \endfirsthead % Header for the first page

    % No \endhead content needed

    % No \endfoot content needed

    \hline % Footer for the last page
    \endlastfoot

    % --- Table Content ---
    \textbf{Luồng cơ bản} & 1. Người dùng truy cập tab tin nhắn. \newline
                           2. Người dùng bấm vào một cuộc hội thoại muốn gửi tin nhắn. \newline
                           3. Hệ thống hiển thị thông tin của cuộc hội thoại và các tin nhắn trong cuộc hội thoại đó. \newline
                           4. Người dùng nhập tin nhắn muốn gửi. \newline
                           5. Người dùng bấm gửi. \newline
                           6. Hệ thống gửi tin nhắn và hiển thị tin nhắn vừa gửi lên cuộc hội thoại. \\
    \hline
    \textbf{Luồng thay thế} & \textbf{Đính kèm địa điểm:} \newline
                               1. Người dùng nhấn nút ``+'' cạnh ô input. \newline
                               2. Hệ thống hiển thị giao diện tìm kiếm/chọn địa điểm. \newline
                               3. Người dùng chọn một địa điểm. \newline
                               4. Hệ thống thêm thông tin địa điểm đã chọn vào nội dung tin nhắn đang soạn thảo. \\
    \hline
    \textbf{Tiền điều kiện} & - Người dùng đang đăng nhập và phiên đăng nhập chưa kết thúc. \newline
                           - Người dùng đã có ít nhất một cuộc hội thoại. \\
    \hline
    \textbf{Hậu điều kiện} & - Hệ thống lưu tin nhắn vào cơ sở dữ liệu và hiển thị tin nhắn trong cuộc hội thoại trong thời gian thực. \newline
                           - Hệ thống nhận diện địa điểm trong tin nhắn và highlight các địa điểm đó. \newline
                           - Hệ thống phân loại tin nhắn có cần thiết cho tổng hợp lịch trình hay không. \newline
                           - Người dùng có thể nhấn vào địa điểm trong tin nhắn để xem chi tiết địa điểm. \newline
                           - Người dùng có thể react hoặc gỡ tin nhắn. \\
    \hline
    \textbf{Yêu cầu phi chức năng} & Hệ thống xử lý gửi tin nhắn dưới 1s. \\
    % --- End Table Content ---

\end{longtable}
\vspace{0.8cm}

\begin{table}[H] % Wrap the diagrams table
    \centering
    \caption{Biểu đồ hoạt động và quan hệ ca sử dụng gửi tin nhắn} % Add caption
    \label{tab:uc_send_message_diagrams} % Add label
    \begin{tabular}{| c | c |}
        \hline
        \textbf{Biểu đồ hoạt động} & \textbf{Quan hệ} \\
        \hline
        \includegraphics[width=0.5\linewidth]{figures/c3/3-3-9-ad.png} % Specified width
        &
        \includegraphics[width=0.45\linewidth]{figures/c3/3-3-9-rd.png} \\ % Specified width
        \hline
    \end{tabular}
\end{table}

\begin{figure}[H]
    \centering
    \includegraphics[width=1\textwidth]{figures/c3/3-3-9-sd.png} % Specified width
    \caption{Biểu đồ tuần tự ca sử dụng gửi tin nhắn.}
    \label{fig:3-3-9-sequence-diagram}
\end{figure}
\section{Các mô hình, thư viện và cơ sở lý thuyết}
Bên cạnh các công nghệ nền tảng về hệ thống đã trình bày, việc triển khai các tính năng thông minh và tự động hóa trong VieVu dựa trên việc ứng dụng các mô hình học máy, thư viện chuyên dụng và các khái niệm lý thuyết liên quan trong lĩnh vực Trí tuệ Nhân tạo (AI) và Học máy (ML). Phần này sẽ giới thiệu cơ sở về các thành phần AI/ML cốt lõi được sử dụng, tập trung vào các kỹ thuật Xử lý Ngôn ngữ Tự nhiên (NLP) cho việc hỗ trợ tạo lịch trình và các phương pháp trong Hệ thống gợi ý (Recommendation Systems) để đề xuất địa điểm.

\subsection{Thư viện Underthesea}

Underthesea~\cite{underthesea_lib} là một thư viện được phát triển bằng ngôn ngữ Python, đóng vai trò là một bộ công cụ mã nguồn mở NLP toàn diện, cung cấp các mô hình tiền huấn luyện và giải pháp dễ tiếp cận cho cộng đồng xử lý tiếng Việt. Thư viện này hỗ trợ nhiều tác vụ NLP nền tảng, ví dụ như phân tách từ (Word Segmentation), gán nhãn từ loại (POS Tagging), và Nhận dạng thực thể tên (Named Entity Recognition - NER), giúp nhanh chóng tích hợp khả năng xử lý tiếng Việt vào ứng dụng. Trong ứng dụng VieVu, chức năng NER từ thư viện \texttt{underthesea} được sử dụng để phân tích nội dung tin nhắn trong các nhóm chat, qua đó tự động trích xuất tên các thực thể Địa điểm (Location - LOC) được người dùng đề cập, phục vụ cho việc thu thập thông tin hỗ trợ tạo lịch trình.

\subsection{Mô hình pre-trained XLM RoBERTA cho ứng dụng phân loại Zero-Shot}

VieVu sử dụng mô hình \texttt{joeddav/xlm-roberta-large-xnli}~\cite{xlm_roberta_xnli_model}, một mô hình có sẵn và phổ biến trên nền tảng Hugging Face, thường được sử dụng qua thư viện \texttt{transformers}. Mô hình này được xây dựng dựa trên kiến trúc XLM-RoBERTa~\cite{xlm_roberta_paper}, một phiên bản cải tiến của RoBERTa với khả năng hiểu và xử lý hiệu quả nhiều ngôn ngữ. Điểm đặc biệt của mô hình \texttt{joeddav/xlm-roberta-large-xnli} là nó đã được tinh chỉnh (fine-tuned) trên bộ dữ liệu XNLI (Cross-lingual Natural Language Inference). Quá trình tinh chỉnh này huấn luyện cho mô hình khả năng thực hiện tốt nhiệm vụ Suy luận Ngôn ngữ Tự nhiên (NLI) - tức là xác định mối quan hệ logic (kéo theo, mâu thuẫn, trung lập) giữa một cặp câu được đưa vào.
 
 Chính việc được huấn luyện trên NLI cho phép mô hình này được ứng dụng một cách hiệu quả cho kỹ thuật phân loại Zero-Shot (Zero-Shot Classification - ZSC). Mặc dù không được huấn luyện trực tiếp để phân loại tin nhắn thành các lớp như "travel schedule" hay "not travel schedule" trong VieVu, mô hình có thể thực hiện điều này bằng cách diễn giải bài toán phân loại dưới dạng bài toán NLI. Cụ thể, tin nhắn cần phân loại được xem là "premise", và mỗi lớp mục tiêu được diễn đạt thành một "hypothesis" (ví dụ: "Tin nhắn này là về địa điểm"). Mô hình sẽ đánh giá mức độ "kéo theo" (entailment) giữa tin nhắn và từng giả thuyết, từ đó xác định lớp phù hợp nhất cho tin nhắn đó mà không cần dữ liệu huấn luyện có nhãn riêng cho các lớp này.
% \textit{Ứng dụng trong VieVu: Hệ thống dùng mô hình này để phân loại tin nhắn chat dựa trên mức độ liên quan đến việc lập kế hoạch chuyến đi, lọc ra các tin nhắn hữu ích cho việc tạo lịch trình.}
% \subsection{Mô hình Gemini 2.0 Flash}
% \label{subsec:gemini_model}

% \textbf{Gemini}~\cite{gemini_google_blog} là một dòng các mô hình ngôn ngữ lớn (Large Language Models - LLMs) đa phương thức (multimodal) được phát triển bởi Google DeepMind. Các mô hình Gemini được thiết kế để hiểu và xử lý thông tin từ nhiều loại dữ liệu đầu vào khác nhau, bao gồm văn bản, hình ảnh, âm thanh và video. Khả năng đa phương thức này cho phép Gemini thực hiện các tác vụ phức tạp đòi hỏi sự kết hợp và suy luận trên nhiều nguồn thông tin.

% Phiên bản \textbf{Gemini 2.0 Flash} là một trong các mô hình thuộc dòng Gemini, được tối ưu hóa cho tốc độ và hiệu quả, phù hợp cho các ứng dụng cần phản hồi nhanh và xử lý lượng lớn yêu cầu. Mặc dù được tối ưu về tốc độ, Gemini 2.0 Flash vẫn duy trì khả năng hiểu ngôn ngữ tự nhiên mạnh mẽ, bao gồm khả năng sinh văn bản, tóm tắt, trả lời câu hỏi, và hiểu ngữ cảnh phức tạp.

% Trong hệ thống VieVu, mô hình Gemini 2.0 Flash được sử dụng làm nền tảng cho tính năng tổng hợp lịch trình từ hội thoại. Hệ thống gửi các đoạn hội thoại của người dùng (đã qua tiền xử lý) cùng với các chỉ dẫn (prompt) chi tiết đến Gemini API. Mô hình sẽ phân tích nội dung, nhận diện các ý định, địa điểm, thời gian và các thông tin liên quan khác để tự động tạo ra một bản nháp lịch trình dưới dạng cấu trúc JSON mà ứng dụng có thể sử dụng trực tiếp. Khả năng hiểu ngữ cảnh và tuân theo chỉ dẫn phức tạp của Gemini là yếu tố then chốt giúp tính năng này hoạt động hiệu quả.


% \subsection{Hệ thống gợi ý (Recommendation Systems)}
% Hệ thống gợi ý (Recommendation Systems) là các thuật toán và kỹ thuật được thiết kế để đề xuất các đối tượng phù hợp nhất cho người dùng cụ thể, dựa trên dữ liệu về người dùng, đối tượng, và tương tác giữa chúng. VieVu áp dụng hai phương pháp tiếp cận nền tảng là Gợi ý dựa trên nội dung và Lọc cộng tác.
 
%  \textbf{Gợi ý dựa trên nội dung (Content-Based Filtering - CB)\cite{cb_concept}} 
%  Phương pháp này đề xuất đối tượng mới cho người dùng dựa trên sự tương đồng về đặc điểm nội dung của những đối tượng mà người dùng đó đã thể hiện sự yêu thích trong quá khứ. Hệ thống sẽ phân tích các thuộc tính, đặc trưng của đối tượng (ví dụ: thể loại, mô tả, từ khóa của địa điểm), xây dựng một hồ sơ phản ánh sở thích của người dùng qua nội dung các đối tượng đã tương tác, và sau đó tìm kiếm, đề xuất những đối tượng mới có nội dung phù hợp nhất với hồ sơ này.
 
% %  \textit{Ứng dụng trong VieVu: Dùng để gợi ý các địa điểm du lịch có nội dung liên quan đến địa điểm người dùng đang xem.}
 
%  \textbf{Lọc cộng tác (Collaborative Filtering - CF)\cite{cf_concept}} 
%  Phương pháp này đưa ra gợi ý bằng cách khai thác hành vi và sở thích của cộng đồng người dùng, thể hiện qua ma trận tương tác người dùng-đối tượng (ví dụ: ma trận ratings). Thay vì phân tích nội dung, hệ thống tìm kiếm các mẫu tương đồng trong cách người dùng tương tác với các đối tượng. Nó có thể xác định những người dùng có cùng sở thích, cùng đánh giá trong quá khứ hoặc những đối tượng thường được cộng đồng yêu thích cùng nhau, từ đó dự đoán và đề xuất những đối tượng mà người dùng hiện tại có khả năng sẽ thích. Các kỹ thuật như phân rã ma trận (Matrix Factorization) thường được sử dụng để khám phá các mối liên hệ tiềm ẩn này từ dữ liệu tương tác.
% %  \textit{VieVu áp dụng các nguyên tắc của CF làm cơ sở cho hệ thống gợi ý địa điểm chính, chủ yếu dựa trên dữ liệu đánh giá (ratings) của cộng đồng người dùng.}
% \subsection{Neural Network cho hệ thống gợi ý}


% Mạng Nơ-ron nhân tạo (Neural Networks - NN) là một lớp các mô hình học máy được lấy cảm hứng từ cấu trúc và hoạt động của não bộ con người, đặc biệt là khả năng học hỏi từ dữ liệu~\cite{nn_deep_learning_book}. Về cơ bản, một mạng NN bao gồm nhiều đơn vị xử lý đơn giản gọi là nơ-ron (neurons), được tổ chức thành các lớp (layers): lớp đầu vào (input layer) nhận dữ liệu, một hoặc nhiều lớp ẩn (hidden layers) thực hiện các phép biến đổi tính toán, và lớp đầu ra (output layer) đưa ra kết quả dự đoán. Các nơ-ron trong các lớp liền kề được kết nối với nhau bằng các liên kết có trọng số (weights). Quá trình học của mạng NN diễn ra bằng cách điều chỉnh các trọng số này để tối thiểu hóa một hàm mất mát (loss function), tức là giảm thiểu sự khác biệt giữa dự đoán của mô hình và kết quả thực tế trên dữ liệu huấn luyện.

% % Một trong những sức mạnh lớn nhất của NN, đặc biệt là các kiến trúc Học Sâu (Deep Learning) với nhiều lớp ẩn, là khả năng **tự động học và trích xuất các đặc trưng (feature extraction)** phức tạp và có ý nghĩa từ dữ liệu đầu vào thô. Thay vì phải dựa vào việc thiết kế đặc trưng thủ công (manual feature engineering), NN có thể tự khám phá ra các biểu diễn dữ liệu (data representations) ở các mức độ trừu tượng khác nhau trong các lớp ẩn của nó. Các biểu diễn này, thường là các vector dày đặc gọi là **vector biểu diễn nhúng (embedding vectors)**, nắm bắt được các thông tin cốt lõi và các mối quan hệ tiềm ẩn trong dữ liệu.

% Trong lĩnh vực Hệ thống gợi ý, khả năng này của NN được ứng dụng rộng rãi để mô hình hóa các tương tác phức tạp giữa người dùng và đối tượng. NN có thể học các vector embedding cho người dùng và đối tượng từ ma trận tương tác (ví dụ: ratings), mã hóa sở thích của người dùng và thuộc tính của đối tượng vào các không gian vector tiềm ẩn. Sự phù hợp giữa một người dùng và một đối tượng sau đó có thể được dự đoán bằng cách kết hợp (ví dụ: qua phép nhân vô hướng hoặc các lớp mạng nơ-ron khác) các embeddings tương ứng của chúng. Đối với các đặc điểm dạng hạng mục (categorical features), ví dụ như loại hình địa điểm du lịch, kỹ thuật One-Hot Encoding (OHE)~\cite{ohe_concept} thường được sử dụng để chuyển đổi chúng thành dạng vector nhị phân mà NN có thể xử lý.

% Hệ thống gợi ý địa điểm chính của ứng dụng VieVu được xây dựng dựa trên một mô hình Neural Network Hybrid. Mô hình này được huấn luyện để học từ dữ liệu đánh giá (dù là giả lập) và đồng thời kết hợp các đặc điểm chi tiết của địa điểm (trong đó có các loại hình đã được mã hóa OHE) và đặc trưng người dùng, nhằm cung cấp các đề xuất được cá nhân hóa.

% % ...existing code...
\subsection{Mô hình Gemini 2.0 Flash}
\label{subsec:gemini_model}

Gemini~\cite{gemini_google_blog} là một dòng mô hình ngôn ngữ lớn (LLM) đa phương thức từ Google DeepMind, có khả năng xử lý văn bản, hình ảnh, âm thanh và video. Phiên bản Gemini 2.0 Flash được tối ưu cho tốc độ và hiệu quả, phù hợp với các tác vụ đòi hỏi phản hồi nhanh mà vẫn duy trì khả năng hiểu ngôn ngữ tự nhiên mạnh mẽ.

Trong VieVu, Gemini 2.0 Flash là nền tảng cho tính năng tổng hợp lịch trình từ hội thoại. Hệ thống gửi các đoạn hội thoại (đã tiền xử lý) và chỉ dẫn (prompt) đến Gemini API. Mô hình phân tích nội dung, nhận diện ý định, địa điểm, thời gian để tự động tạo bản nháp lịch trình dạng JSON, hỗ trợ bởi khả năng hiểu ngữ cảnh và tuân theo chỉ dẫn phức tạp.

\subsection{Hệ thống gợi ý (Recommendation Systems)}
Hệ thống gợi ý (Recommendation Systems) là các thuật toán và kỹ thuật được thiết kế để đề xuất các đối tượng phù hợp nhất cho người dùng cụ thể, dựa trên dữ liệu về người dùng, đối tượng, và tương tác giữa chúng. VieVu áp dụng hai phương pháp tiếp cận nền tảng là Gợi ý dựa trên nội dung và Lọc cộng tác.
 
Gợi ý dựa trên nội dung (Content-Based Filtering - CB)\cite{cb_concept}
 Phương pháp này đề xuất đối tượng mới cho người dùng dựa trên sự tương đồng về đặc điểm nội dung của những đối tượng mà người dùng đó đã thể hiện sự yêu thích trong quá khứ. Hệ thống sẽ phân tích các thuộc tính, đặc trưng của đối tượng (ví dụ: thể loại, mô tả, từ khóa của địa điểm), xây dựng một hồ sơ phản ánh sở thích của người dùng qua nội dung các đối tượng đã tương tác, và sau đó tìm kiếm, đề xuất những đối tượng mới có nội dung phù hợp nhất với hồ sơ này.
 
%  \textit{Ứng dụng trong VieVu: Dùng để gợi ý các địa điểm du lịch có nội dung liên quan đến địa điểm người dùng đang xem.}
 
Lọc cộng tác (Collaborative Filtering - CF)\cite{cf_concept}
 Phương pháp này đưa ra gợi ý bằng cách khai thác hành vi và sở thích của cộng đồng người dùng, thể hiện qua ma trận tương tác người dùng-đối tượng (ví dụ: ma trận ratings). Thay vì phân tích nội dung, hệ thống tìm kiếm các mẫu tương đồng trong cách người dùng tương tác với các đối tượng. Nó có thể xác định những người dùng có cùng sở thích, cùng đánh giá trong quá khứ hoặc những đối tượng thường được cộng đồng yêu thích cùng nhau, từ đó dự đoán và đề xuất những đối tượng mà người dùng hiện tại có khả năng sẽ thích. Các kỹ thuật như phân rã ma trận thường được sử dụng để khám phá các mối liên hệ tiềm ẩn này từ dữ liệu tương tác.
%  \textit{VieVu áp dụng các nguyên tắc của CF làm cơ sở cho hệ thống gợi ý địa điểm chính, chủ yếu dựa trên dữ liệu đánh giá (ratings) của cộng đồng người dùng.}
\subsection{Neural Network cho hệ thống gợi ý}
Mạng Nơ-ron nhân tạo (Neural Networks - NN) là mô hình học máy lấy cảm hứng từ não bộ, gồm các nơ-ron tổ chức thành lớp (đầu vào, ẩn, đầu ra) kết nối bằng trọng số. NN học bằng cách điều chỉnh trọng số để tối thiểu hóa hàm mất mát, tức là giảm sai lệch giữa dự đoán và thực tế.

Một ưu điểm lớn của NN, đặc biệt là Học Sâu, là khả năng tự động học và trích xuất đặc trưng phức tạp từ dữ liệu thô, tạo ra các vector biểu diễn nhúng (embedding vectors) nắm bắt thông tin cốt lõi.

Trong Hệ thống gợi ý, NN mô hình hóa tương tác người dùng-đối tượng, học vector embedding cho người dùng và đối tượng từ ma trận tương tác, mã hóa sở thích và thuộc tính vào không gian vector tiềm ẩn. Sự phù hợp được dự đoán bằng cách kết hợp các embeddings này. Đặc điểm hạng mục (ví dụ: loại hình địa điểm) thường được chuyển đổi bằng One-Hot Encoding (OHE)~\cite{ohe_concept}.

Hệ thống gợi ý địa điểm của VieVu sử dụng mô hình Neural Network Hybrid, huấn luyện từ dữ liệu đánh giá (giả lập) và kết hợp đặc điểm địa điểm (đã OHE) cùng đặc trưng người dùng để cung cấp đề xuất cá nhân hóa.
% ...existing code...
\subsection{Cosine Similarity cho hệ thống gợi ý}


Độ tương đồng Cosine (Cosine Similarity)~\cite{cosine_similarity_concept} là một phép đo được sử dụng để định lượng mức độ giống nhau giữa hai đối tượng (cụ thể là các địa điểm du lịch) dựa trên biểu diễn vector của chúng. Về bản chất, Cosine Similarity đo lường cosin của góc giữa hai vector trong một không gian tích trong (inner product space). Giá trị này cho biết liệu hai vector có đang hướng về cùng một phía một cách tương đối hay không; giá trị càng gần 1, hai vector càng có hướng tương đồng (góc giữa chúng nhỏ), ngụ ý rằng các đối tượng mà chúng biểu diễn có nhiều điểm chung. Cosine Similarity thường được ứng dụng rộng rãi trong phân tích văn bản để đo độ tương đồng giữa các tài liệu. Công thức tính được cho bởi:
$$\text{Cosine Similarity}(\vec{a}, \vec{b}) = \frac{\vec{a} \cdot \vec{b}}{\|\vec{a}\| \|\vec{b}\|}$$

Để áp dụng Cosine Similarity vào việc so sánh nội dung văn bản của các địa điểm, nội dung đó trước tiên cần được chuyển đổi thành dạng vector số học. Kỹ thuật Bag-of-Words (BoW)~\cite{bow_concept} là một phương pháp nền tảng để thực hiện việc này. BoW biểu diễn một đoạn văn bản (như mô tả địa điểm) thành một vector, trong đó mỗi chiều của vector tương ứng với một từ trong một bộ từ vựng (vocabulary) đã được xác định trước. Mô hình BoW bỏ qua trật tự từ và cấu trúc ngữ pháp, chỉ tập trung vào các từ xuất hiện.


\section{Các công nghệ nền tảng hệ thống} % Sử dụng \section thay vì \subsection
\label{sec:core_technologies}

Phần này giới thiệu tổng quan về các công nghệ chính được sử dụng để xây dựng các thành phần cốt lõi của hệ thống VieVu, bao gồm giao diện người dùng (frontend), logic xử lý phía máy chủ (backend) và hệ thống lưu trữ dữ liệu (database).

\subsection{Flutter} % Sử dụng \subsection thay vì \subsubsection
\label{subsec:frontend_flutter}

Giao diện người dùng của ứng dụng VieVu được phát triển hoàn toàn bằng Flutter~\cite{flutter_doc}, một bộ công cụ (UI toolkit) mã nguồn mở được phát triển bởi Google. Flutter sử dụng ngôn ngữ lập trình Dart và cho phép xây dựng ứng dụng di động hoạt động trên cả hai nền tảng Android và iOS từ một cơ sở mã nguồn duy nhất (single codebase). Lựa chọn này không chỉ giúp đảm bảo trải nghiệm người dùng đồng nhất trên các nền tảng mà còn tăng tốc độ phát triển nhờ tính năng hot reload và thư viện widget phong phú, tùy biến cao. Về mặt kiến trúc, mã nguồn phía client tuân thủ theo các nguyên tắc của Clean Architecture~\cite{clean_arch} với cách tổ chức thư mục theo Feature-First, nhằm đảm bảo tính rõ ràng, dễ bảo trì và kiểm thử. Việc quản lý trạng thái (state management) trong ứng dụng được xử lý linh hoạt thông qua thư viện Bloc/Cubit (\texttt{flutter\_bloc})~\cite{bloc_package}, một giải pháp phổ biến giúp tách biệt logic nghiệp vụ khỏi giao diện người dùng hiệu quả.

\subsection{Supabase (BaaS)} % Sử dụng \subsection
\label{subsec:backend_supabase}

Để xử lý các nghiệp vụ cốt lõi và quản lý dữ liệu phía máy chủ, VieVu tận dụng nền tảng Supabase~\cite{supabase_doc}, một giải pháp Backend as a Service (BaaS) mã nguồn mở xây dựng trên nền tảng PostgreSQL. Việc sử dụng Supabase giúp tăng tốc độ phát triển nhờ cung cấp sẵn các dịch vụ backend thiết yếu. Các tính năng chính được khai thác bao gồm hệ thống cơ sở dữ liệu PostgreSQL mạnh mẽ để lưu trữ dữ liệu quan hệ (người dùng, chuyến đi, địa điểm,...); tính năng Realtime thông qua cơ chế Pub/Sub để đồng bộ dữ liệu trực tiếp cho chat và chia sẻ vị trí; hệ thống Authentication tích hợp sẵn để quản lý định danh và xác thực người dùng an toàn (hỗ trợ nhiều phương thức); khả năng thực thi tác vụ nền định kỳ (Scheduled Tasks) thông qua Cron Jobs (\texttt{pg\_cron}) và PostgreSQL Functions; cùng với cơ chế bảo mật Row Level Security (RLS) để kiểm soát quyền truy cập dữ liệu chi tiết ở cấp độ hàng.

\subsection{Python và FastAPI} % Sử dụng \subsection
\label{subsec:backend_fastapi}

Bên cạnh Supabase, một API server riêng biệt được xây dựng bằng ngôn ngữ Python và framework FastAPI~\cite{fastapi_doc} để phục vụ các tác vụ xử lý phức tạp và chuyên biệt. FastAPI là một web framework Python hiện đại, hiệu năng cao, hỗ trợ lập trình bất đồng bộ, rất phù hợp để xây dựng các API RESTful. Server này chịu trách nhiệm chính trong việc cung cấp các API endpoint cho các chức năng ứng dụng AI/ML (như gợi ý địa điểm, phân loại tin nhắn, tổng hợp lịch trình) và thực hiện việc tra cứu, tích hợp dữ liệu từ các API của bên thứ ba (như Trip.com, Ticketbox).

\subsection{PostgreSQL} % Sử dụng \subsection
\label{subsec:database_postgres} % Label mới để tránh trùng

Hệ quản trị cơ sở dữ liệu (CSDL) trung tâm của VieVu là PostgreSQL~\cite{postgresql_doc}, một hệ quản trị CSDL quan hệ - đối tượng (ORDBMS) mã nguồn mở được quản lý thông qua nền tảng Supabase. PostgreSQL chịu trách nhiệm lưu trữ toàn bộ dữ liệu quan hệ cốt lõi của ứng dụng, bao gồm thông tin người dùng, chuyến đi, địa điểm, lịch trình, đánh giá và các mối quan hệ giữa chúng. Việc lựa chọn PostgreSQL phù hợp nhờ độ tin cậy cao, mô hình dữ liệu quan hệ mạnh mẽ và tuân thủ tốt chuẩn ACID. Khả năng mở rộng của PostgreSQL thông qua các tính năng SQL nâng cao như PostgreSQL Functions cũng được tận dụng cho các tác vụ nền tự động. Sự tích hợp chặt chẽ với các dịch vụ khác của Supabase như Realtime và RLS càng củng cố vai trò của PostgreSQL như một nền tảng dữ liệu vững chắc cho VieVu.

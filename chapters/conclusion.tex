\chapter*{Kết luận}
% \label{ch:conclusion_final_revised}
\addcontentsline{toc}{chapter}{Kết luận} % Thêm vào mục lục nếu muốn

Khóa luận này đã trình bày quá trình xây dựng và đánh giá ứng dụng di động VieVu, một giải pháp hỗ trợ lập kế hoạch du lịch nhóm tích hợp AI. Hệ thống được triển khai thành công sử dụng Flutter cho client và kiến trúc backend với Supabase BaaS và Python/FastAPI, cung cấp các chức năng cốt lõi từ quản lý chuyến đi, lịch trình đến tương tác nhóm. Điểm nhấn công nghệ của VieVu là việc tích hợp các thành phần AI: pipeline NLP tự động đề xuất lịch trình từ chat (ứng dụng NER, ZSC, LLM) và hệ thống gợi ý địa điểm (kết hợp Content-Based và Neural Network Hybrid). Các tính năng AI này, đặc biệt là khả năng tổng hợp lịch trình, mang lại giá trị thiết thực, giúp các thành viên trong nhóm tránh rối loạn thông tin và hỗ trợ người mới tham gia nhanh chóng nắm bắt kế hoạch chung.

Quá trình triển khai, tuy đạt được mục tiêu, đã gặp nhiều khó khăn do thiếu kinh nghiệm và kiến thức về mảng AI/ML trong dự án thực tế. Một trong những khó khăn và cũng là điểm đáng tiếc nhất là việc thiếu hụt dữ liệu rating thực tế từ người dùng. Giải pháp tạm thời phải đưa ra là tạo dữ liệu rating giả lập dựa trên thống kê của địa điểm để huấn luyện mô hình gợi ý NN. Việc này dẫn đến kết quả chênh lệch dự đoán đánh giá (MAE $\approx$ 0.64~\ref{fig:mae}) so với thực tế, làm giảm đi độ tin cậy và khả năng cá nhân hóa thực sự của gợi ý so với tiềm năng nếu có dữ liệu thật, khiến kết quả chỉ có thể dừng ở mức tạm ổn. Bên cạnh đó, pipeline tạo lịch trình có sự phụ thuộc đáng kể vào LLM (Gemini) cho các tác vụ phức tạp và mô hình NER không quá chính xác.

Từ những hạn chế này, em đã định hướng được rõ ràng cho các bước phát triển tiếp theo của hệ thống. Ưu tiên hàng đầu là xây dựng cơ chế thu thập dữ liệu rating thực tế để cải thiện căn bản chất lượng gợi ý. Đồng thời, cần nghiên cứu tinh chỉnh một mô hình chuyên biệt cho việc tạo lịch trình nhằm giảm sự phụ thuộc vào LLM và nâng cao độ chính xác của các thành phần tiền xử lý như NER. Các nghiên cứu người dùng chi tiết cũng cần được thực hiện để đánh giá trải nghiệm thực tế.

Tóm lại, khóa luận đã hoàn thành mục tiêu đề ra, xây dựng thành công nền tảng ứng dụng VieVu với các tính năng AI sáng tạo, giải quyết bài toán thực tế trong lập kế hoạch du lịch nhóm. Dù còn những điểm cần hoàn thiện xuất phát từ những khó khăn và giới hạn trong quá trình thực hiện, nhưng thông qua khóa luận này, em đã tích lũy được nhiều kiến thức và kinh nghiệm quý báu trong lĩnh vực phát triển ứng dụng di động cũng như AI/ML và hy vọng sẽ có thể hoàn thành các định hướng đề ra cho VieVu trong tương lai gần.

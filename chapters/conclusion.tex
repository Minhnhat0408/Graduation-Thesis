\chapter*{Kết luận}
\addcontentsline{toc}{chapter}{Kết luận}

\noindent Khóa luận này đã trình bày quá trình nghiên cứu, thiết kế và triển khai hệ thống ứng dụng di động VieVu, một giải pháp hỗ trợ người dùng kết nối, lập kế hoạch và chia sẻ trải nghiệm du lịch nhóm. Xuất phát từ nhu cầu thực tế về một công cụ tích hợp và thông minh, dự án đã xây dựng một nền tảng kết hợp các chức năng khám phá thông tin, quản lý lịch trình, tương tác xã hội và ứng dụng trí tuệ nhân tạo.

% Về mặt kỹ thuật, VieVu được triển khai thành công với kiến trúc kết hợp linh hoạt giữa BaaS (Supabase) cho các tác vụ cơ bản và backend tùy chỉnh (Python/FastAPI) cho các xử lý phức tạp và AI. Ứng dụng client Flutter đảm bảo trải nghiệm đa nền tảng, cùng cơ sở dữ liệu PostgreSQL được thiết kế tối ưu.

Các chức năng cốt lõi đã được hiện thực hóa, bao gồm khám phá và tìm kiếm thông tin du lịch, quản lý chuyến đi và lịch trình chi tiết, tương tác xã hội qua nhắn tin thời gian thực, và ứng dụng AI để phân loại tin nhắn, tóm tắt hội thoại thành lịch trình, và gợi ý địa điểm cá nhân hóa. Quá trình triển khai đã chứng minh tính khả thi của kiến trúc và hiệu quả của các giải pháp công nghệ. Việc tích hợp AI vào lập kế hoạch nhóm thông qua phân tích hội thoại tự động là điểm nhấn quan trọng, mang lại sự tiện lợi đáng kể. Kiến trúc xử lý tin nhắn hai giai đoạn cũng đảm bảo cân bằng giữa cập nhật liên tục và hiệu năng.

Tuy nhiên, hệ thống vẫn còn một số hạn chế như bộ dữ liệu cần mở rộng, các mô hình AI cần tinh chỉnh và có dữ liệu về đánh giá thực tế sâu hơn với dữ liệu lớn, và quá trình kiểm thử cần mở rộng để đảm bảo độ ổn định.

Hướng phát triển trong tương lai cho VieVu bao gồm: (1) Thu thập dữ liệu rating thực tế từ người dùng để cải thiện chất lượng hệ thống gợi ý. (2) Tinh chỉnh và đánh giá sâu hơn các mô hình AI, đặc biệt là pipeline tạo lịch trình, có thể thử nghiệm fine-tuning các mô hình ngôn ngữ. (3) Mở rộng thêm các tính năng xã hội và tương tác cộng đồng. (4) Phát triển phiên bản web cho ứng dụng. (5) Thực hiện các nghiên cứu người dùng (user studies) để đánh giá trải nghiệm và thu thập phản hồi chi tiết.

Tóm lại, luận văn đã hoàn thành mục tiêu xây dựng một hệ thống ứng dụng di động hỗ trợ lập kế hoạch du lịch nhóm tích hợp AI. VieVu giải quyết các bài toán thực tế và mở ra hướng tiếp cận mới bằng công nghệ BaaS và AI. Dù còn điểm cần cải thiện, nền tảng VieVu đã vững chắc, sẵn sàng cho các bước phát triển tiếp theo để trở thành công cụ hữu ích cho cộng đồng du lịch Việt Nam.
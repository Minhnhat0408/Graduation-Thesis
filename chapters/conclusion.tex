\chapter*{Kết luận}
% \label{ch:conclusion_final_revised}
\addcontentsline{toc}{chapter}{Kết luận} % Thêm vào mục lục nếu muốn
\noindent Khóa luận này đã trình bày việc xây dựng thành công ứng dụng di động VieVu, cung cấp một giải pháp tích hợp cho việc lập kế hoạch và quản lý chuyến đi du lịch nhóm. Hệ thống đáp ứng các chức năng cơ bản như quản lý chuyến đi, quản lý lịch trình, khám phá địa điểm, tương tác nhóm qua chat, cùng với đó là những tính năng nâng cao ứng dụng AI như hệ thống gợi ý địa điểm và đặc biệt là khả năng tự động đề xuất lịch trình từ nội dung hội thoại.

Bằng việc tích hợp các chức năng lập kế hoạch, giao tiếp và ứng dụng AI vào một nền tảng duy nhất, VieVu giải quyết được sự bất tiện và xung đột thường gặp trong quá trình lên kế hoạch du lịch nhóm. Tính năng tự động đề xuất lịch trình từ tin nhắn giúp giảm thiểu thời gian tổng hợp thủ công, hạn chế xung đột ý kiến và cho phép các thành viên (đặc biệt là người mới) nhanh chóng nắm bắt kế hoạch chung. Các hệ thống gợi ý cũng góp phần tăng khả năng khám phá, đề xuất các lựa chọn phù hợp hơn với người dùng. Kiến trúc backend kết hợp giữa Supabase và FastAPI cũng chứng tỏ sự linh hoạt, cho phép vừa tận dụng BaaS để tăng tốc phát triển, vừa có không gian riêng để triển khai các giải pháp AI phức tạp.

Tuy nhiên, quá trình triển khai cũng đối mặt với những thách thức nhất định, chủ yếu liên quan đến ứng dụng kỹ thuật AI/ML và các vấn đề về dữ liệu. Khó khăn trong việc thu thập dữ liệu từ nguồn bên ngoài đã đòi hỏi phải xây dựng giải pháp kỹ thuật riêng biệt. Đặc biệt, sự thiếu hụt dữ liệu đánh giá thực tế là một trở ngại lớn, buộc nhóm tác giả phải sử dụng dữ liệu giả lập dựa trên thống kê địa điểm để huấn luyện mô hình gợi ý chính. Dù mô hình này cho kết quả hoạt động ban đầu (MAE $\approx$ 0.64~\ref{fig:mae}), % <<< NHỚ THAY LABEL ĐÚNG
việc sử dụng dữ liệu giả lập là một hạn chế cần thừa nhận, có thể làm giảm độ tin cậy và tính cá nhân hóa thực sự của gợi ý. Bên cạnh đó, tính năng tự động tạo lịch trình hiện còn phụ thuộc nhiều vào mô hình ngôn ngữ lớn bên ngoài (Gemini). Do đó, hướng phát triển trong tương lai cần ưu tiên việc thu thập dữ liệu đánh giá thực tế để cải thiện hệ thống gợi ý, đồng thời nghiên cứu huấn luyện một mô hình AI chuyên biệt hơn cho việc tạo lịch trình.

Tóm lại, khóa luận đã hoàn thành mục tiêu đề ra, xây dựng được một hệ thống VieVu có tiềm năng ứng dụng cao, giải quyết bài toán thực tế trong lĩnh vực du lịch nhóm bằng các công nghệ hiện đại và AI. Dù còn những điểm cần cải thiện xuất phát từ những khó khăn và giới hạn khách quan, nền tảng được xây dựng là vững chắc, mở ra nhiều hướng phát triển giá trị trong tương lai.
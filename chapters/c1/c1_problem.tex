\section{Giới thiệu bài toán}

Ngày nay, du lịch ngày càng trở thành một phần không thể thiếu trong đời sống hiện đại.
Theo số liệu từ Bộ Văn hóa, Thể thao và Du lịch, năm 2024, Việt Nam đón khoảng 17,5 triệu lượt khách quốc tế, tăng 38.9\% so với năm 2023~\cite{bvhttdl2024}, cho thấy nhu cầu du lịch đang tăng
trưởng mạnh mẽ. Cùng với đó, xu hướng du lịch nhóm ``chữa lành'' ngày càng phổ biến, đặc biệt trong giới trẻ và các gia đình, nhờ vào mong muốn chia sẻ trải nghiệm, khám phá và nghỉ ngơi.\nl
\indent Tuy nhiên, việc lập kế hoạch cho một chuyến đi nhóm không hề đơn giản, đòi hỏi sự phối hợp chặt chẽ giữa các thành viên để thống nhất lịch trình, lựa chọn địa điểm, 
phương tiện di chuyển và các hoạt động giải trí phù hợp. 
Trong thực tế, đa số nhóm du lịch hiện nay vẫn 
dựa vào các nền tảng nhắn tin phổ biến như Zalo, Messenger 
để thảo luận và sắp xếp kế hoạch. Tuy nhiên, 
thông tin quan trọng về thời gian, địa điểm, hoạt động thường bị phân tán trong 
hàng trăm tin nhắn, dẫn đến khó khăn trong việc theo dõi
, điều chỉnh và tổ chức lịch trình. Việc tổng hợp thủ công từ
 các đoạn hội thoại này không chỉ mất thời gian mà còn
  dễ bỏ sót những chi tiết quan trọng. \nl
\indent Hiện nay, trên thị trường có một số ứng dụng hỗ trợ du lịch, nhưng hầu hết tập trung vào đặt vé, khách sạn, hoặc gợi ý địa điểm hơn là hỗ trợ lập kế hoạch chuyến đi một cách toàn diện. Các công cụ lập kế hoạch hiện có thường đơn giản, không hỗ trợ tốt cho việc tạo lịch trình nhóm hoặc tìm kiếm bạn đồng hành. Do đó, một hệ thống hỗ trợ tổ chức và quản lý chuyến đi nhóm
, cho phép người dùng tương tác, trao đổi thông tin trực tiếp, 
đồng thời tự động tổng hợp lịch trình từ hội thoại nhóm,
 sẽ là một giải pháp cần thiết. Hệ thống này không chỉ tiện lợi trong lập kế hoạch mà còn 
 cải thiện trải nghiệm du lịch nhóm, 
 giúp người dùng kết nối dễ dàng hơn với cộng đồng du lịch.\nl

\section{Hiện trạng trên thị trường}
Tại Việt Nam, nhiều nền tảng du lịch trực tuyến đã được phát triển nhằm đáp ứng nhu cầu du lịch ngày càng tăng. Các nền tảng này chủ yếu do doanh nghiệp trong nước hoặc các công ty quốc tế triển khai, cung cấp các dịch vụ từ đặt vé, khách sạn đến gợi ý lịch trình. Dưới đây là một số ứng dụng tiêu biểu:

\begin{itemize}
    \item[-]\textbf{TripHunter}~\cite{triphunter}: Cung cấp dịch vụ tìm kiếm thông tin điểm đến, đặt vé máy bay, khách sạn và các dịch vụ du lịch khác. Ngoài ra, ứng dụng hỗ trợ tạo lịch trình tự động và dự toán chi phí cho chuyến đi.

    \item[-]\textbf{Trip.com}~\cite{tripcom}: Nền tảng du lịch trực tuyến toàn cầu, hỗ trợ đặt vé máy bay, khách sạn, thuê xe và các tour du lịch. Ứng dụng cung cấp dịch vụ đa ngôn ngữ và đa tiền tệ, phù hợp với du khách quốc tế.

    \item[-]\textbf{TripAdvisor}~\cite{tripadvisor}: Một trong những nền tảng đánh giá du lịch lớn nhất thế giới, cung cấp đánh giá, ý kiến và hình ảnh về các điểm đến du lịch. TripAdvisor giúp người dùng lên kế hoạch dựa trên đánh giá cộng đồng và có hỗ trợ tạo lịch trình tự động.

    % \item[-]\textbf{TripIt}~\cite{tripit}: Ứng dụng hỗ trợ người dùng tổ chức và quản lý kế hoạch du lịch bằng cách tự động tạo lịch trình từ các email xác nhận đặt chỗ.

\end{itemize}

Mặc dù các nền tảng trên có nhiều ưu điểm về giao diện và dữ liệu, chúng chủ yếu tập trung vào việc cung cấp thông tin, đặt dịch vụ cho cá nhân hoặc tạo lịch trình dựa trên đề xuất khách quan. Các tính năng hỗ trợ đặc thù cho việc lập kế hoạch và phối hợp trong chuyến đi nhóm – một xu hướng du lịch quan trọng – vẫn còn hạn chế và chưa được tối ưu hóa. Điều này dẫn đến thực trạng phổ biến là người dùng vẫn phải dựa vào các nền tảng nhắn tin riêng lẻ để thảo luận, khiến thông tin bị phân tán và việc tổng hợp kế hoạch trở nên thủ công, tốn thời gian và dễ sai sót, đi ngược lại mong muốn về một công cụ quản lý tập trung mà nhiều người dùng mong đợi. % Giữ lại citation nếu muốn

Nhận thấy những hạn chế đó, ứng dụng VieVu được đề xuất như một giải pháp chuyên biệt hơn cho du lịch nhóm, mang lại những ưu điểm khác biệt. Cụ thể, VieVu giải quyết các vấn đề trên thông qua việc: (1) Tích hợp chặt chẽ việc nhắn tin trao đổi và việc lập, cập nhật lịch trình vào cùng một nền tảng, loại bỏ sự phân mảnh công cụ và đảm bảo thông tin luôn đồng bộ. (2) Hỗ trợ hiệu quả quá trình lập kế hoạch dựa trên sự đồng thuận của nhóm, kết hợp sự tiện lợi của việc tạo lịch trình từ các mục đã lưu với sức mạnh của AI trong việc tự động nhận diện địa điểm và tổng hợp ý kiến từ hội thoại. (3) Tăng cường mạnh mẽ tính cộng đồng và trải nghiệm thực tế trong chuyến đi bằng các tính năng như công khai chuyến đi tìm bạn đồng hành, mời bạn bè, cập nhật trạng thái tự động, chia sẻ vị trí thời gian thực và đánh giá chuyến đi/thành viên sau khi kết thúc.

Với những giải pháp tích hợp và thông minh này, VieVu hướng tới việc nâng cao đáng kể trải nghiệm lập kế hoạch và tham gia du lịch nhóm, tạo ra một môi trường kết nối hiệu quả hơn so với các ứng dụng hiện có trên thị trường.
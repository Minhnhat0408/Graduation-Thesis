\section{Hiện trạng trên thị trường}
\subsection{Thị trường Việt Nam}

Tại Việt Nam, nhiều nền tảng du lịch trực tuyến đã được phát triển nhằm đáp ứng nhu cầu du lịch ngày càng tăng. Các nền tảng này chủ yếu do doanh nghiệp trong nước hoặc các công ty quốc tế triển khai, cung cấp các dịch vụ từ đặt vé, khách sạn đến gợi ý lịch trình. Dưới đây là một số ứng dụng tiêu biểu:

\begin{itemize}
    \item[-]\textbf{TripHunter}~\cite{triphunter}: Cung cấp dịch vụ tìm kiếm thông tin điểm đến, đặt vé máy bay, khách sạn và các dịch vụ du lịch khác. Ngoài ra, ứng dụng hỗ trợ tạo lịch trình tự động và dự toán chi phí cho chuyến đi.

    \item[-]\textbf{Trip.com}~\cite{tripcom}: Nền tảng du lịch trực tuyến toàn cầu, hỗ trợ đặt vé máy bay, khách sạn, thuê xe và các tour du lịch. Ứng dụng cung cấp dịch vụ đa ngôn ngữ và đa tiền tệ, phù hợp với du khách quốc tế.

    \item[-]\textbf{TripAdvisor}~\cite{tripadvisor}: Một trong những nền tảng đánh giá du lịch lớn nhất thế giới, cung cấp đánh giá, ý kiến và hình ảnh về các điểm đến du lịch. TripAdvisor giúp người dùng lên kế hoạch dựa trên đánh giá cộng đồng và có hỗ trợ tạo lịch trình tự động.

    % \item[-]\textbf{TripIt}~\cite{tripit}: Ứng dụng hỗ trợ người dùng tổ chức và quản lý kế hoạch du lịch bằng cách tự động tạo lịch trình từ các email xác nhận đặt chỗ.

\end{itemize}

Mặc dù các nền tảng này có lợi thế về giao diện thân thiện, dữ liệu địa điểm du lịch phong phú và khả năng hỗ trợ đặt dịch vụ, nhưng chúng chủ yếu tập trung vào đặt vé, khách sạn và tour du lịch. Các tính năng liên quan đến lập kế hoạch chuyến đi nhóm hay kết nối mọi người vẫn chưa được chú trọng cũng như chưa tối ưu hóa quy trình tổ chức du lịch nhóm, khiến người dùng vẫn phải dựa vào các nền tảng nhắn tin để thảo luận và tự tổng hợp thông tin một cách thủ công.


\subsection{Điểm mạnh so với các ứng dụng trên thị trường}

Trên thị trường du lịch trong nước, hầu hết các ứng dụng hiện nay chủ yếu tập trung vào việc cung cấp thông tin đặt vé, khách sạn hay tạo lịch trình dựa trên đề xuất cá nhân. Một số nền tảng như TripHunter và TripAdvisor mặc dù có hỗ trợ tạo lịch trình tự động bằng AI, nhưng đều chỉ giới hạn ở khía cạnh cá nhân hóa cho mỗi người dùng, mà chưa khai thác triệt để nhu cầu lập kế hoạch chuyến đi nhóm – một xu hướng ngày càng phổ biến trong du lịch hiện nay.

Do đó, ứng dụng VieVu được đề xuất nhằm khắc phục các hạn chế của những công cụ hiện có đối với việc tổ chức du lịch nhóm. Cụ thể, VieVu mang lại các giải pháp chính:

\textbf{Tích hợp nhắn tin và lập kế hoạch trong một nền tảng duy nhất:}
Cho phép người dùng vừa trao đổi thông tin qua chat vừa cập nhật, theo dõi lịch trình chuyến đi theo thời gian thực mà không cần chuyển đổi giữa nhiều ứng dụng hay tổng hợp thủ công. 

\textbf{Hỗ trợ hiệu quả việc lập kế hoạch nhóm dựa trên ý kiến các thành viên:}
Thay vì chỉ dựa vào AI hay mẫu có sẵn. Hệ thống đơn giản hóa quy trình này bằng cách cho phép tạo lịch trình từ các mục đã lưu, tự động nhận diện địa điểm trong tin nhắn và tổng hợp ý kiến nhóm bằng AI để tạo bản nháp lịch trình.

\textbf{Tăng cường tính cộng đồng và trải nghiệm thực tế:}
Thông qua việc cho phép công khai chuyến đi tìm bạn đồng hành, mời bạn bè, tự động cập nhật trạng thái chuyến đi, chia sẻ vị trí thời gian thực khi đi, và đánh giá chuyến đi cũng như các thành viên sau khi kết thúc.

Với những tính năng trên, VieVu không chỉ nâng cao trải nghiệm lập kế hoạch du lịch cho người dùng mà còn tạo ra một môi trường kết nối cộng đồng hiệu quả, khác biệt so với các ứng dụng hiện có trên thị trường.



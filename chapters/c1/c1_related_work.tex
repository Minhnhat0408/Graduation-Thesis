\section{Hiện trạng trên thị trường}
\subsection{Thị trường Việt Nam}

Tại Việt Nam, nhiều nền tảng du lịch trực tuyến đã được phát triển nhằm đáp ứng nhu cầu du lịch ngày càng tăng. Các nền tảng này chủ yếu do doanh nghiệp trong nước hoặc các công ty quốc tế triển khai, cung cấp các dịch vụ từ đặt vé, khách sạn đến gợi ý lịch trình. Dưới đây là một số ứng dụng tiêu biểu:

\begin{itemize}
    \item \textbf{TripHunter}\cite{triphunter}: Cung cấp dịch vụ tìm kiếm thông tin điểm đến, đặt vé máy bay, khách sạn và các dịch vụ du lịch khác. Ngoài ra, ứng dụng hỗ trợ tạo lịch trình tự động và dự toán chi phí cho chuyến đi.

    \item \textbf{MyTour}\cite{mytour}: Chuyên về dịch vụ đặt phòng khách sạn với nhiều ưu đãi và hỗ trợ khách du lịch trong việc tìm kiếm chỗ ở phù hợp.

    \item \textbf{Trip.com}\cite{tripcom}: Nền tảng du lịch trực tuyến toàn cầu, hỗ trợ đặt vé máy bay, khách sạn, thuê xe và các tour du lịch. Ứng dụng cung cấp dịch vụ đa ngôn ngữ và đa tiền tệ, phù hợp với du khách quốc tế.

    \item \textbf{TripAdvisor}\cite{tripadvisor}: Một trong những nền tảng đánh giá du lịch lớn nhất thế giới, cung cấp đánh giá, ý kiến và hình ảnh về các điểm đến du lịch. TripAdvisor giúp người dùng lên kế hoạch dựa trên đánh giá cộng đồng và có hỗ trợ tạo lịch trình tự động.

    \item \textbf{TripIt}\cite{tripit}: Ứng dụng hỗ trợ người dùng tổ chức và quản lý kế hoạch du lịch bằng cách tự động tạo lịch trình từ các email xác nhận đặt chỗ.

\end{itemize}

Mặc dù các nền tảng này có lợi thế về giao diện thân thiện, dữ liệu địa điểm du lịch phong phú và khả năng hỗ trợ đặt dịch vụ, nhưng chúng chủ yếu tập trung vào đặt vé, khách sạn và tour du lịch. Các tính năng liên quan đến lập kế hoạch chuyến đi nhóm hay kết nối mọi người vẫn chưa được chú trọng cũng như chưa tối ưu hóa quy trình tổ chức du lịch nhóm, khiến người dùng vẫn phải dựa vào các nền tảng nhắn tin để thảo luận và tự tổng hợp thông tin một cách thủ công.


\subsection{Điểm mạnh so với các ứng dụng trên thị trường}

Trên thị trường du lịch trong nước, hầu hết các ứng dụng hiện nay chủ yếu tập trung vào việc cung cấp thông tin đặt vé, khách sạn hay tạo lịch trình dựa trên đề xuất cá nhân. Một số nền tảng như TripHunter, TripIt và TripAdvisor mặc dù có hỗ trợ tạo lịch trình tự động bằng AI, nhưng đều chỉ giới hạn ở khía cạnh cá nhân hóa cho mỗi người dùng, mà chưa khai thác triệt để nhu cầu lập kế hoạch chuyến đi nhóm – một xu hướng ngày càng phổ biến trong du lịch hiện nay.

Do đó, ứng dụng VieVu được đề xuất không chỉ kế thừa các ưu điểm của các hệ thống hiện có, mà còn khắc phục các hạn chế liên quan đến du lịch nhóm thông qua:
\begin{itemize}
  \item \textbf{Tích hợp nhắn tin và tổ chức chuyến đi trong một nền tảng}: Thay vì phải sử dụng nhiều ứng dụng khác nhau như Zalo hay Messenger để trao đổi thông tin và sau đó tổng hợp thủ công, VieVu cho phép người dùng trò chuyện trực tiếp và cập nhật lịch trình chuyến đi của mình theo thời gian thực.

  \item \textbf{Hỗ trợ lập kế hoạch chuyến đi nhóm}: Đơn giản hóa quy trình lập kế hoạch chuyến đi vì có thể lưu trữ tra cứu các điểm đến và tạo lịch trình ngay từ các mục lưu đó, đồng thời tích hợp tính năng tự nhận diện các địa điểm trong tin nhắn và tổng hợp ý kiến từ các thành viên trong nhóm thông qua AI/NLP, giúp tạo lịch trình dựa trên thảo luận trực tiếp trong nhóm.

  \item \textbf{Tăng tính cộng đồng}: Người dùng có thể dễ dàng công khai chuyến đi để tìm kiếm bạn đồng hành và mời bạn bè tham gia. Hệ thống sẽ tự động cập nhật trạng thái chuyến đi theo tiến độ lịch trình và cho phép người dùng chia sẻ vị trí khi chuyến đi diễn ra, cũng như đánh giá chuyến đi và bạn đồng hành của mình sau khi chuyến đi kết thúc, góp phần xây dựng một cộng đồng du lịch tin cậy và thúc đẩy cải thiện chất lượng các chuyến đi tương lai.

\end{itemize}
Với những tính năng trên, VieVu không chỉ nâng cao trải nghiệm lập kế hoạch du lịch cho người dùng mà còn tạo ra một môi trường kết nối cộng đồng hiệu quả, khác biệt so với các ứng dụng hiện có trên thị trường.


